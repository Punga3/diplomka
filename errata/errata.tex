\documentclass{article}
\usepackage{inputenc}[utf-8]
\usepackage{babel}[english]
\usepackage{indentfirst}
\usepackage{amsmath}
\usepackage{amssymb}
\usepackage{amsthm}
\usepackage[a-2u]{pdfx}

\def\llbracket{[\![}
\def\rrbracket{]\!]}
\newcommand{\Th}{\text{Th}}
\newcommand{\st}{\text{st}}
\newcommand{\bbl}{\llbracket}
\newcommand{\bbr}{\rrbracket}
\newcommand{\Lor}{\bigvee}
\newcommand{\Land}{\bigwedge}
\newcommand{\1}{\textbf{1}}
\newcommand{\0}{\textbf{0}}
\newcommand{\dpt}{\text{dp}}
\newcommand{\Aut}{\text{Aut}}
\newcommand{\pPATH}{*\text{PATH}}
\newcommand{\pDPATH}{*\text{DPATH}}
\newcommand{\pPATHl}{*\text{PATH}^\leq}
\newcommand{\EDGE}{\text{EDGE}}
\newcommand{\nEDGE}{\text{nonEDGE}}
\newcommand{\ALL}{\text{ALL}}
\newcommand{\leqp}{\leq_l}
\newcommand{\set}[1]{\langle #1)}
\newcommand{\ceil}[1]{\left \lceil {#1} \right \rceil}
\newcommand{\floor}[1]{\left \lfloor {#1} \right \rfloor}
\newcommand{\M}{\mathcal{M}}
\newcommand{\Po}{\mathcal{P}}
\newcommand{\A}{\mathcal{A}}
\newcommand{\B}{\mathcal{B}}
\newcommand{\I}{\mathcal{I}}
\newcommand{\card}{\text{card}}
\newcommand{\SK}{\text{SK}}
\newcommand{\CK}{\text{CK}}
\newcommand{\G}{\mathcal{G}}
\newcommand{\Q}{\mathbb{Q}}
\newcommand{\Z}{\mathbb{Z}}
\newcommand{\T}{\mathcal{T}}
\renewcommand{\S}{\mathcal{S}}
\renewcommand{\P}{\textbf{P}}
\newcommand{\NP}{\textbf{NP}}
\newcommand{\FP}{\textbf{FP}}
\newcommand{\TFNP}{\textbf{TFNP}}
\newcommand{\LEAF}{\text{LEAF}}
\newcommand{\SOURCEORSINK}{\text{SOURCE.OR.SINK}}
\newcommand{\PPA}{\textbf{PPA}}
\newcommand{\PPAD}{\textbf{PPAD}}
\newcommand{\N}{\mathbb{N}}
\newcommand{\dom}{\text{dom}\:}
\newcommand{\Rng}{\text{Rng}\:}
\newcommand{\Frac}{\text{Frac}}


\begin{document}
\begin{center}
Ondřej Ježil, Pseudofinite structures and limits

\normalsize
\textbf{ERRATA}
\end{center}

\textbf{p. 10} The first paragraph on the page (cont. of \textbf{Definition 1.4.1}) is missing a quantification for the parameter $t$. It should read: 

``We define $\T_{rud}$ to be the set of all $(T,\ell)$ of depth at most $g_n^{1/t}$, for some nonstandard $t\in\M$, and $F_{rud}$ to be the set of all functions computed by some $(T,\ell)\in \T_{rud}$. For brevity, we will leave the labeling of the trees out of the notation so a tree in $\T_{rud}$ can be denoted just by $T$.''

\vspace{0.5em}
\textbf{p. 12, proof of Theorem 2.1.1.} The first sentence in the proof is poorly formulated. Let us restate it as follows.

``
Any two trees $T_\alpha$ and $T_\beta$ computing potential witnesses $\alpha,\beta$ of the formula $E(x,y)$ on some subset of $\Omega$ can be combined then into one tree that outputs an edge on the same subset of $\Omega$, so we can just analyze the case where the witnesses are computed by the same tree.''

\vspace{0.5em}
\textbf{p. 16, Example 2.3.2.} 
The number of edges should be $\ceil{\frac{k(k-1)}{2\log k}}$.


\vspace{0.5em}
\textbf{p. 17, Theorem 2.3.4.} The statement should read:

``
Let $F$ be any vertex family, $\G_k$ a wide sequence and let $\varphi_0(\overline x)$ be an open $\{E\}$-formula such that
\[\lim_{k\to\infty}\Pr_{G\in\mathcal{G}_k}[G\models(\forall \overline x)\varphi_0(\overline x)]=1.\]

Then $\lim_F \G_n \bbl(\forall \overline x)\varphi_0(\overline x)\bbr=\1.$''

\vspace{0.5em}
\textbf{p. 18, Example 2.3.7.} 
The number of edges should be $\ceil{\frac{k(k-1)}{2\log k}}$.

\vspace{0.5em}
\textbf{p. 19, Theorem 2.4.2.} The explanation of the notation $\psi^b$ is missing. For an $\{E\}$-formula $\psi$ we have
\[\psi^b:=
\begin{cases}
\psi & b=1\\
\lnot \psi & b=0.\\
\end{cases}
\]

\vspace{0.5em}
\textbf{p. 20, Corollary 2.4.3.} ``\ldots we have that $\bbl(\forall\overline y)(\exists \overline x)\varphi(\overline x,\overline y)\bbr=\1.$''

\vspace{0.5em}
\textbf{p. 28, Conjecture 3.2.2.} Let us stress that the parameter $m$ is an element of $\M$.

\vspace{0.5em}
\textbf{p. 32, proof of Corollary 4.1.7.} The inequalities contain a typo and a numerical error, here is the corrected version. The rest of the proof is unaffected.
\begin{align*}
\Pr_{G\in\G_{n-c}}[\text{$T$ fails}]&\geq \prod_{i=0}^{n^{1/t}}\left(1-\frac{2}{n-2i-c-2}\right)\tag{4.26}\\
&\geq \left(1-\frac{2(n^t+1)}{n-2n^{1/t}-c-2}\right)\tag{4.27}
\end{align*}

\vspace{0.5em}
\textbf{p. 32, proof of Theorem 4.1.8.} There is a duplication error and the proof should be read from the beginning of page \textbf{33}.

\end{document}
