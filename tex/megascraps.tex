
\chapter*{Scraps}
\begin{defi}
We say that $\{\mathcal{G}_k\}_{k=0}^\infty$ is \textbf{categorical} if there is $k_0$ such that for every $k>k_0$ if we have $G_1,G_2\in\mathcal{G}_k$ then $G_1\cong G_2$. For a categorical sequence $\{G_k\}_{k=0}^\infty$ we denote $G_k$ the lexicographically minimal element of $\mathcal{G}_k$.
\end{defi}

\begin{lemm}
Let $\{\mathcal{G}_k\}_{k=0}^\infty$ be categorical and isomorphism closed, then for large enough $k$
\[\abs{\mathcal{G}_k}=\frac{g_k!}{\abs{\Aut(G_k)}}.\]
\end{lemm}
\begin{proof}
Every $\rho\in S_{g_k}$ defines an isomorphism $\rho:G_k\to\rho(G_k)$, where $\rho(G_k)$ is a graph obtained from $G_k$ by renaming every vertex $v$ to $\rho(v)$.

\textbf{Claim:} For any $\rho,\pi\in S_{g_k}$:
\[\rho(G_k)=\pi(G_k)\iff\exists \tau\in\Aut(G_k):\rho\circ\tau=\pi.\]
\textit{Proof of claim.} ``$\Rightarrow$" Let $\rho(G_k)=\pi(G_k)$, therefore $\tau:=\rho^{-1}\circ\pi\in\Aut(G_k)$ and $\rho\circ\tau=\rho\circ\rho^{-1}\circ\pi=\pi.$

``$\Leftarrow$" Let $\rho\circ\tau=\pi$. Then $\pi(G_k)=\rho(\tau(G_k))=\rho(G_k). \qedhere$

Notice that the $\tau$ in the statement of the claim is uniquely determinted by $\rho^{-1}\circ\pi$. Therefore if we defined a quotient set $S_{g_k}/\sim$ with $\rho\sim\pi\iff\rho(G_k)=\pi(G_k)$ then $\abs{S_{g_k}/\sim}=\frac{g_k!}{\abs{\Aut(G_k)}}$.

The Lemma follows from noticing that if we start with $\{G_k\}$ and then we build $\mathcal{G}_k$ by finding isomorphic graphs on the vertex set $[g_k]$ we can only do so by trying different permutation from $S_{g_k}$ and these permutations find the same graph if and only if they are in the same $\sim$-class. Therefore there is a bijection between $S_{g_k}/\sim$ and $\mathcal{G}_k$. \qed
\end{proof}

\begin{lemm}[Candidate for optimal search trees]
Let $\{\mathcal{G}_k\}_{k=0}^\infty$ be categorical and isomorphism closed, let $\varphi(x_0,\dots,x_{l-1})$ be an open $\{E\}$-formula, let $\models \varphi(\overline x) \to \Land_{i,j=0}^{l-1,l-1}x_i=^{b_{ij}}x_j$ for some $b_{ij}\in\{0,1\}$, let $k_0\geq0$ and define $\{q_k\}_{k=k_0}^\infty$ as follows

\[q_k:=\frac{g_k!}{\abs{\Aut(G_k)}}\cdot\frac{\abs{\varphi(G_k)}}{\abs{\bigcup_{G\in\mathcal{G}}\varphi(G_k)}}.\]

Then there is $c\in\mathbb{N}$ and trees $T_0,\dots,T_{l-1}$ of depth $n^{(r)} \cdot c$, (with $n^{(r)}$ being defined in the proof) such that for the $\overline \alpha$ computed by $\overline T$ we have $\bbl\varphi(\overline \alpha)\bbr=\1$.
\end{lemm}

\begin{proof}
We will use the identity from the statement to construct a search tree (iterated $T_{\overline a}$) which almost always finds a witness to $\varphi$.

We will analyze the problem in the finite case for big enough $k>0$. We should only check those tuples included in $\bigcup_{G\in\mathcal{G}_k}\varphi(G)$. For example, if we are trying to find an edge then we need not check the constant tuples $(a,a)$. Moreover, to succeed we only need to check one specific tuple in each $\varphi(G),G\in\mathcal{G}_k$.

Consider the set $S=\{(G,\overline a); G\in \mathcal{G}_k, G\models\varphi(\overline a)\}$ and a projection to the second coordinate $p_2: S\to\bigcup_{G\in\mathcal{G}_k}\varphi(G)$. Since $\abs{S}=\frac{g_k!}{\abs{\Aut(G)}}\cdot\abs{\varphi(G_k)}$ we have that $q_k$ is the average size of a $p_2$ preimage of any $\overline a\in \bigcup_{G\in\mathcal{G}_k}\varphi(G)$. 

\textbf{Claim:} For all $\overline a,\overline b\in\bigcup_{G\in\mathcal{G}_k}\varphi(G)$ we have $\abs{p_2^{-1}[\overline a]}=\abs{p_2^{-1}[\overline b]}=q_k$.

\textit{Proof of claim.} We will prove that for any $\overline a,\overline b\in\bigcup_{G\in\mathcal{G}_k}\varphi(G)$ we have $\abs{p_2^{-1}[\overline a]}\leq\abs{p_2^{-1}[\overline b]}$, by symmetry, they must be equal and also equal to $q_k$ which is the average size of any singleton preimage.

Let $p_2^{-1}[\overline a]=\{G_0,\dots,G_{s-1}\}\times\{\overline a\}$ and let $\rho=(b_0\: a_0)\dots(b_{l-1}\:a_{l-1})$, this is a permutation from the condition on $\varphi$. Then \[p_2^{-1}[\overline b]\supseteq\{\rho(G_0),\dots,\rho(G_{s-1})\}\times\{\rho(\overline a)=\overline b\}.\qed\] 

Now consider the multiset $M=(\bigcup_{G\in\mathcal{G}_k}\varphi(G),\text{count}:\overline a \mapsto \abs{p_2^{-1}[\overline a]})$, we will construct the searching tree by plucking elements from this multiset in the following way.

Let $M^{(0)}:=M,\mathcal{G}_k^{(0)}=\mathcal{G}_k$. For $i\geq0$ and $M^{(i)},\mathcal{G}_k^{(i)}$ built, take some $\overline a\in M^{(i)}$ with maximal $\text{count}(\overline a)$, put $\mathcal{G}_k^{(i+1)}=\mathcal{G}_k^{(i)}\setminus p_2^{-1}[\overline a]$ and form $M^{(i+1)}$ by removing $\overline a$, and for every $\overline b \in p_1[p_2^{-1}[\overline a]]\setminus\{\overline a\}$ setting $\text{count}_{M^{(i+1)}}(\overline b)=\max\{0,\text{count}_{M^{(i)}}(\overline b)-(\varphi(G_k))\}$. We also add $T_{\overline a}$ to the leaves of the tree we are constructing $T_i$ and call it $T_{i+1}$.

For each $i\geq 0$ we have that $T_i$ finds a witness in $G\in\mathcal{G}_k$ iff $G\not\in\mathcal{G}_k^{(i)}$. So to calculate the probability of success of $T_i$ we just need to find upper bounds on the cardinality of $\mathcal{G}_k^{(i)}$. 

Define $m_i:=\max\{\text{count}(\overline a);\overline a \in M^{(i)}\}$. Let $k^{(0)}\geq 0$ be the greatest number such that for all $i<k^{(0)}$: 
\(m_i=q_k.\) 

Define a set $M_m^{(i)}=\{\overline a;\text{count}_{M_i}(\overline a)=m_i\}$. We can see, that $k^{(0)}\geq1$ and $M^{(0)}=\bigcup_{G\in\mathcal{G}_k}\varphi(G)$. At each step $i<k^{(0)}$ we construct $T_{i+1}$ by searching for some $\overline a \in M_{m}^{(i)}$, this results in $\abs{\mathcal{G}_k^{(i+1)}}=\abs{\mathcal{G}_k^{i}}-q_k$. We also remove one instance of every $\overline b \in p_1[p_2^{-1}(\overline a)]\setminus\{\overline a\}$ from $M^{(i)}$ to form $M^{(i+1)}$, this results in $\abs{M^{(i+1)}_m}\geq \abs{M^{(i)}_m}-1-q_k\cdot(\abs{\varphi(G_k)-1}).$

Therefore 

\begin{align}
k^{(0)}&\geq \left \lfloor \frac{\abs{M^{(0)}_m}}{q_k\cdot\abs{\varphi(G_k)}} \right \rfloor\\
&=
\left\lfloor\frac{\abs{\bigcup_{G\in\mathcal{G}_k}\varphi(G)}}{q_k\cdot\abs{\varphi(G_k)}}\right\rfloor
,
\end{align}

and $\abs{\mathcal{G}_k^{(k^{(0)})}}=\abs{\mathcal{G}_k}-k^{(0)}\cdot q_k=\frac{\abs{\Aut(G_k)}}{g_k!}-\left\lfloor\frac{\abs{\cup_{G\in\mathcal{G}_k}\varphi(G)}}{q_k\cdot\abs{\varphi(G_k)}}\right\rfloor\cdot q_k\leq \frac{\abs{\Aut(G_k)}}{g_k!}-\left\lfloor\frac{\abs{\cup_{G\in\mathcal{G}_k}\varphi(G)}}{\abs{\varphi(G_k)}} \right \rfloor.$

However the right hand side of the last inequality is rarely $\leq 0$, so generally one has to continue with plucking even after $k^{(0)}$-many steps. We define $k^{(j)}$ as the greatest number such that for all $i<k^{(j)}:m_i\geq q_k-j$ and continue for $k^{(r)}$ steps, where $r$ is the smallest number such that


\begin{align}
\abs{\mathcal{G}_{k}^{k^{(r)}}}&=\abs{\mathcal{G}_k}-k^{(0)}\cdot q_k- \sum_{j=1}^{r}(k^{(j)}-k^{(j-1)})\cdot (q_k - j)\\
&=0.
\end{align}

However, this requires a general analysis of $k^{(j)}$ and I haven't manage to compute that.

For $k=n$ in $\mathcal{M}$ we put $n^{(r)}:=k^{(r)}.$
\end{proof}

\begin{lemm}
Let $\G_k$ be a isomorphism closed categorical wide sequence, let $\varphi(\overline x)$ be an open $\{E\}$-sentence, then there exists a tree $T$ of depth
\[\floor{\frac{\abs{\bigcup_{G\in\G_n}\varphi(G)}^2}{\abs{\G_k}\abs{\varphi(G_n)}^2}}\]
such that
\[\Pr_{G\in G_k}[G\models \varphi^{\Gamma}(T(G))]\geq \frac{\abs{\bigcup_{G\in\G_k}\varphi(G)}}{\abs{\G_k}\abs{\varphi(G_k)}}-\frac{\abs{\varphi(G_k)}}{\abs{\bigcup_{G\in\G_n}\varphi(G)}}.\]
\end{lemm}
\begin{proof}
We will use the identity from the statement to construct the tree $T$ (iterated $T_{\overline a}$) which finds a witness with the stated probability.

We will analyze the problem in the finite case for big enough $k>0$. We should only check those tuples included in $\bigcup_{G\in\mathcal{G}_k}\varphi(G)$. For example, if we are trying to find an edge then we need not check the constant tuples $(a,a)$. Moreover, to succeed we only need to check one specific tuple in each $\varphi(G),G\in\mathcal{G}_k$.

Consider the set $S=\{(G,\overline a); G\in \mathcal{G}_k, G\models\varphi(\overline a)\}$ and a projection to the second coordinate $p_2: S\to\bigcup_{G\in\mathcal{G}_k}\varphi(G)$. Since $\abs{S}=\frac{g_k!}{\abs{\Aut(G)}}\cdot\abs{\varphi(G_k)}$ we have that 
\[q_k:=\abs{\G_k}\cdot \frac{\abs{\varphi(G_k)}}{\abs{\bigcup_{G\in\G_k}\varphi(\G_k)}}\] 
is the average size of a $p_2$ preimage of any $\overline a\in \bigcup_{G\in\mathcal{G}_k}\varphi(G)$. 

\textbf{Claim:} For all $\overline a,\overline b\in\bigcup_{G\in\mathcal{G}_k}\varphi(G)$ we have $\abs{p_2^{-1}[\overline a]}=\abs{p_2^{-1}[\overline b]}=q_k$.

\textit{Proof of claim.} We will prove that for any $\overline a,\overline b\in\bigcup_{G\in\mathcal{G}_k}\varphi(G)$ we have $\abs{p_2^{-1}[\overline a]}\leq\abs{p_2^{-1}[\overline b]}$, by symmetry, they must be equal and also equal to $q_k$ which is the average size of any singleton preimage.

Let $p_2^{-1}[\overline a]=\{G_0,\dots,G_{s-1}\}\times\{\overline a\}$ and let $\rho=(b_0\: a_0)\dots(b_{l-1}\:a_{l-1})$, this is a permutation from the condition on $\varphi$. Then \[p_2^{-1}[\overline b]\supseteq\{\rho(G_0),\dots,\rho(G_{s-1})\}\times\{\rho(\overline a)=\overline b\}.\qedhere\] 

Now consider the multiset $M=(\bigcup_{G\in\mathcal{G}_k}\varphi(G),\text{count}:\overline a \mapsto \abs{p_2^{-1}[\overline a]})$, we will construct the searching tree by removing elements from this multiset in the following inductive procedure. Let $M^{0}:=M$ and $\G_k^{0}:=\G_k$.

If you can, take any $\overline a_i \in M^{i}$ with $\abs{p_2^{-1}[a]}=q_k$, otherwise terminate with $l:=i$ and $(\overline a_0,\dots,\overline a_{l-1})$. If we have $\overline a_i$ then construct 
\begin{align}
\G_k^{i+1}&:=\{G\in\G_k^{i};G\not\models \varphi(\overline a) \}\\
S^{i+1}&:=((G,\overline a)\in S; G\in\G_k^{i+1})\\
M^{i+1}&:=(M\setminus \{\overline a\},\text{count:}\overline b \mapsto \abs{p_{2,i}^{-1}[\overline b]})
\end{align}
where $p_{2,i}:S^{i+1}\to\bigcup_{G\in\G_k}\varphi(G)$ is the projection to second second coordinate of $S^{i+1}$.

After the process terminates we are left with a list of tuples $\overline a_0,\dots,\overline a_{l-1}$ such that in all graphs in $\G_k\setminus\G_k^{l}$ at least one of $\overline a_i$s witnesses $\varphi$.

First let us bound $l$. Since in every step we remove $q_k$ graphs from $\G_k^i$ we also lower the $\text{count}$ of at most $\abs{\varphi(G_k)}\cdot q_k$ elements from $M^{i}$, we continue only if there are elements whose count was never lowered so we have
\begin{align}
l&=\floor{\frac{\abs{\bigcup_{G\in\G_k}\varphi(G)}}{q_k\abs{\varphi(G_k)}}}\\
&=\floor{\frac{\abs{\bigcup_{G\in\G_k}\varphi(G)}^2}{\abs{\G_k}\abs{\varphi(G_k)}^2}}.
\end{align}

Now what is the probability that the resulting tree checking for all $\overline a_i$s succeeds in finding a witness? It suceeds on all of $\G_k\setminus\G_k^{l}$ so we have

\begin{align}
\Pr_{G\in G_k}[G\models \varphi^{\Gamma}(T(G))]&=\frac{l\cdot q_k}{\abs{\G_k}}\\
&=\frac{\floor{\frac{\abs{\bigcup_{G\in\G_k}\varphi(G)}}{q_k\abs{\varphi(G_k)}}}\cdot q_k}{\abs{\G_k}}\\
&\geq\frac{\left(\frac{\abs{\bigcup_{G\in\G_k}\varphi(G)}}{q_k\abs{\varphi(G_k)}}-1\right)\cdot q_k}{\abs{\G_k}}\\
&=\frac{\frac{\abs{\bigcup_{G\in\G_k}\varphi(G)}}{\abs{\varphi(G_k)}}-q_k}{\abs{\G_k}}\\
&=\frac{\abs{\bigcup_{G\in\G_k}\varphi(G)}}{\abs{\G_k}\abs{\varphi(G_k)}}-\frac{\abs{\varphi(G_k)}}{\abs{\bigcup_{G\in\G_k}\varphi(G)}},
\end{align}
which proves the lemma.
\qed
\end{proof}

A similar result also helps us understand the behaviour witnessing formulas over an isomorphism closed sequence.

\begin{thrm}
Let $\G_k$ isomorphism closed, let $\varphi_0(\overline x)$ quantifier free and let 
\[\lim_{F_{tree}}\G_n\bbl(\exists \overline x)\varphi_0^\Gamma(\overline x)\bbr=\1.\]

Then in $K(\G_n,F_{rud},G_{rud})$ we have
\[\bbl(\exists F)(F:\set{\floor{n/m}}\hookrightarrow\M)\land(\forall w<\floor{n/m})(\varphi_0^\Gamma(F(w)))\bbr=\1.\]

\end{thrm}

