%%% A template for a simple PDF/A file like a stand-alone abstract of the thesis.

\documentclass[12pt]{report}

\usepackage[a4paper, hmargin=1in, vmargin=1in]{geometry}
\usepackage[a-2u]{pdfx}
\usepackage[utf8]{inputenc}
\usepackage[T1]{fontenc}
\usepackage{lmodern}
\usepackage{textcomp}
\usepackage{amsmath}
\usepackage{amssymb}

\begin{document}

%% Do not forget to edit abstract.xmpdata.

Pro třídu instancí výpočetního problému definujeme limitní objekt, vzhledem k nějaké výpočetně omezené třídě funkcí. Klíčová metoda zde je forcing s náhodnými proměnnými, kde za množinu elementárních jevů bereme instance nestandardní velikosti. Studujeme obecnou teorii těchto limit, v práci nazývaných široké limity, a jejich spojitost s klasickými problémy jako je nalezení velké kliky a nebo s kombinatorickými problémy přidruženými k třídám $\textbf{NP}$ vyhledávacích problémů $\textbf{PPA}$ a $\textbf{PPAD}$. Nášimi hlavními výsledky jsou určité 0-1 zákony pro tyto limity a existence kliky významné velikosti v široké limitě grafů sestávajících z jedné velké kliky.

\end{document}
