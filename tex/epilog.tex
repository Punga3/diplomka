\chapter*{Concluding remarks}
\addcontentsline{toc}{chapter}{Concluding remarks}

In this thesis we built basic theory around wide limits of graphs, proved several general theorems and proved that the wide limits $\lim_F\EDGE_n$, $\lim_{F_{rud}}\ALL_n$ and $\lim_{F_{nbtree}}\pPATH_n$ have complete theories. Moreover we showed that any large portion of a wide sequence with a complete theory has to agree on valid sentences and therefore has also complete theory. We also let a few open problems and conjectures for further research along this way.

During development we planned analyzing wide limits the family $F_{poly}$ of polynomial functions. In \cite{krajicek2010forcing} it was proven that forcing with $F_{poly}$ results is quantifier elimination which implies that if an $\{E\}$-sentence holds in large enough $\G_k$ it has to hold in the limit. However the second order limit can still provide some information about the ability of polynomial time functions to search interesting subsets of $G\in\G_k$. In the end we did not get any new results about it. We want to mention that even though it would seem that $F_{poly}$ limits could depend on the $\P$ vs. $\NP$ question it seems that something a bit different happens. The way the limit objects are defined, it is not enough that some polynomial time algorithm does not exists to see that we cannot witness some property in the limit but it is important that no polynomial algorithm does not work on non-zero fraction of all inputs. This is more close to the generic case polynomial time \cite{gilman2007report}.

Now another question emerges, is there a class $F$ such that the properties of the wide limits relative to it correspond to the $\P$ vs. $\NP$ question? It would have to be some kind of weaker class than $F_{poly}$ whose generic time complexity at least partially corresponds to polynomial time. We did not get to treat this question in any formal way but it indeed seems like an interesting one.

Another natural question would be to consider structures over general languages than just the language of graphs. Other combinatorial structures like hypergraphs and tournaments could be considered. Moreover wide limits of finite universal algebras could be considered which could require a whole new theory. This all leads to the fact that generalized spectra, elementary classes of $\Sigma_1^1$ logic \cite{Fagin74}, with restricted vertex sets make up a wide sequence, there could be a connection to the theory of spectra of sentences.
