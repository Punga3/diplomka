\chapter*{Concluding remarks}
\addcontentsline{toc}{chapter}{Concluding remarks}

In this thesis we built a basic theory around wide limits of graphs, proved several general theorems and described the theories of the wide limits $\lim_F\EDGE_n$, $\lim_{F_{rud}}\ALL_n$, $\lim_{F_{nbtree}}\pPATH_n$, $\lim_{F_{nbtree}}\pPATHl_n$ and $\lim_{F_{dtree}}\pDPATH_n$ and proved that they are complete (Corollary \ref{crlltoosparse}, Theorem \ref{thrmall} and corollaries \ref{crllpPATH}, \ref{crllpPATHl} and \ref{crllpDPATH}). Moreover we showed that any large portion of a wide sequence with a complete theory has to agree on valid sentences and therefore has also complete theory (Lemma \ref{lemmhere}). We also proved that $\lim_{F_{rud}}^{G_{rud}}\SK_n^{1/2}$ contains a clique of size $\floor{n/(2\ln n)}$ (Theorem \ref{thrmsk}) and also that nonexistence of finite cliques in not valid in the more complex wide limit $\lim_{F_{rud}}^{G_{rud}}\CK_n^{1/2}$ (Theorem \ref{thrmck}).

During the development we planned to analyze wide limits the family $F_{poly}$ of polynomial functions. In \cite{krajicek2010forcing} it was proven that forcing with $F_{poly}$ results in quantifier elimination which implies that if an $\{E\}$-sentence holds in large enough $\G_k$, it has to hold in the limit. However the second order limit can still provide some information about the ability of polynomial time functions to search interesting subsets of $G\in\G_k$. In the end we did not get any new results about it. We want to mention that even though it would seem that $F_{poly}$ limits could depend on the $\P$ vs. $\NP$ question it seems that something a bit different happens. The way the limit objects are defined, it is not enough that some polynomial time algorithm does not exist to see that we cannot witness some property in the limit, but it is important that no polynomial algorithm fails on all inputs. This more closely resembles the generic case polynomial time (see \cite{gilman2007report}).

Another natural question would be to consider structures over general languages than just the language of graphs, and allow trees to query atomic sentences. Other combinatorial structures like hypergraphs and tournaments could be considered. Furthermore, wide limits of finite universal algebras could be considered which could require a whole new theory. This all leads to the fact that generalized spectra, elementary classes of $\Sigma_1^1$-logic (see \cite{Fagin74}), with restricted vertex sets make up a wide sequence, there could be a connection to the theory of spectra of sentences.


Considering structures over arbitrary vocabularies opens the door to connecting the concept of a wide limit with the complexity of the Constraint Satisfaction Problem. Solutions to $\text{CSP}(\A)$ form a wide sequence and depending on the tractability of the class we could expect different behaviour from $\lim_{F}\text{CSP}(\A)_n$.

Finally, important direction in studying wide sequences would be to characterize some $\lim_F\G_n$ without the direct construction and therefore to prove upper and lower bounds for $F$.
