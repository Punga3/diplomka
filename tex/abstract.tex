%%% A template for a simple PDF/A file like a stand-alone abstract of the thesis.

\documentclass[12pt]{report}

\usepackage[a4paper, hmargin=1in, vmargin=1in]{geometry}
\usepackage[a-2u]{pdfx}
\usepackage[utf8]{inputenc}
\usepackage[T1]{fontenc}
\usepackage{lmodern}
\usepackage{textcomp}
\usepackage{amsmath}
\usepackage{amssymb}

\begin{document}

%% Do not forget to edit abstract.xmpdata.

For a class of graph instances of a computational problem we define a limit object, relative to some computationally restricted class of functions. The key method here is forcing with random variables where the sample set is taken as instances of some nonstandard size. We study the general theory of these limits, called in the thesis wide limits, and their connection to classical problems such as finding a large clique or with the combinatorial problems associated with the classes of total $\textbf{NP}$ search problems $\textbf{PPA}$ and $\textbf{PPAD}$. Our main results are several 0-1 laws associated with these limits and existence of a significantly large clique of the wide limit of all graph consisting of one large clique.

\end{document}
