\chapter*{Introduction}
\addcontentsline{toc}{chapter}{Introduction}

There exist several logical constructions of limits of classes of finite structures such as the ultraproduct construction and the compactness theorem. The latter was used in \cite{Fagin1976} to prove the 0-1 law for structures over relational vocabularies.

In combinatorics, there are also several notions of limits of finite graphs. For example the dense graph limit defined for a sequence of graphs $\{G_k\}_{k>0}$ satisfying the condition that
\[t(F,G_n)=\frac{\hom(F,G)}{\abs{G_n}^{\abs{F}}}\]
converges for every fixed connected graph $F$ which provided a framework (see \cite{lovasz2006limits}) to restate and find new proofs for results in extremal graph theory. For instance Goodman's theorem relating the number of edges to the number of triangles in a graph. There are other notions of limits of sequences of graphs and we recommend to read \cite{Nesetril2013} to the interested reader. Another recent use of limit objects for the results of extremal combinatorics was by Razborov in \cite{razborov2007flag}.

These different notions of limits directly or tangentially relate to the concept of pseudofinite structures. For a first order language $L$ we call an $L$-structure $S$ pseudofinite if it satisfies the theory $T_f$ consisting of all sentences true in all finite $L$-structures. Of course, the interesting case is when $S$ is itself not finite.

In this thesis we use the concept of pseudofinite structures to define a limit of a family of finite graphs relative to some computationally restricted class of functions $F$. Instead of studying the density of substructures, we study these wide limits (as we shall call them) both generally and by analyzing concrete examples and tying them with the computational complexity of search problems for $F$. The image to keep in mind is that we take a limit of a class of inputs to a specific problem and the shape of the limits reflects how some computationally restricted viewer may see a generic input.

The key method we use is arithmetical forcing with random variables, developed in \cite{krajicek2010forcing}, which allows us to construct models of (weak) arithmetical theories and by restricting to a language of graphs gives us Boolean valued graphs. In these Boolean valued graphs, witnessing of existential quantifiers corresponds to the ability of $F$ to solve search problems over the class of graphs we are considering.

After recalling important concepts in the Preliminaries chapter we define the wide limit in Chapter \ref{chapwidelimit}. In Chapter \ref{chapgeneraltheory} we consider some examples and build around them a general theory. In chapters \ref{chapdense} and \ref{chapsparse} we analyze more complex examples which correspond to the complexity of finding a large clique and to semantic subclasses of $\TFNP$ respectively. 
