\chapter{Forcing with random variables and the wide limit}\label{chapwidelimit}
\section{Setup}

Our goal in this chapter is to provide a definition of a limit of an infinite class of finite graphs in which arbitrarily large graphs occur and the number of graphs in each cardinality tends to infinity. The following definition makes our requirements of such a class of graphs precise. From now on $E$ denotes a binary relational symbol, so that formulas in the theory of graphs are precisely $\{E\}$-sentences.

\begin{defi}
Let $\{\mathcal{G}_k\}_{k>0}$ be a sequence of non-empty finite sets of (directed or undirected) finite graphs, i.e. structures of in the first order language $\{E\}$. We call it a \textbf{wide sequence} if the following holds.

\begin{itemize}
\item There is a strictly increasing sequence of positive whole numbers $\{g_k\}_{k>0}$ such that the underlying set of each $G\in\mathcal{G}_k$ is $\set{g_k}$.
\item $\lim_{k\to\infty} \abs{\mathcal{G}_k}=\infty$
\end{itemize}

By abuse of notation we will denote the wide sequence just $\G_k$.
\end{defi}

The second condition guarantees that $\mathcal{G}_n$ is an infinite set for $n>\mathbb{N}$. In fact, the condition is not strictly important to proceed with this chapter, but it more closely describes what sort of sequences we are interested in. 

We will sometimes talk somewhat loosely about wide limits of a class of finite graphs $\mathcal{C}$ which abbreviates that we imagine the class to be stratified into levels $\mathcal{C}_k=\{G\in\mathcal{C}; V_G=\set{g_k}\}$ determined by some canonical choice of cardinalities. If this sequence is wide we can proceed as if we started with a wide sequence in the first place. Many interesting classes of graphs form a wide sequence in this sense. For example graphs with exactly one edge, graphs with bounded degree and so on. We will get to explore many examples in depth after we define the wide limit. 

\vspace{1em}
The limit is defined by specifying a nonstandard prolongation $\{\tilde \G_n\}_{n<t}$ of the original wide sequence and some nonstandard $n < t$. In the next section we will consider a very rich sublanguage of $L_{all}$ which still makes every wide sequence definable in $\M$ and therefore there is a unique nonstandard prolongation which we will denote just $\{\G_t\}_{t\in\M}$ because it is in fact also unbounded.

In the first order case we will define a Boolean valued graph $\lim_F\G_n$ through which we can investigate the first order properties of the limits. We then define its `arithmetical expansion' $K(\G_n,F)$ which interprets all relational symbols from $L_{all}$. 

We shall be also interested in second order properties of the Boolean valued graph,
for example whether it contains of a large clique.  For this we expand
$\lim_F\G_n$ to $\lim_F^G\G_n$ and $K(\G_n,F)$ to $K(\G_n,F,G)$ with $G$ being second order objects (relations, functions). Here the arithmetical expansion is essential, because in the (second order) language of graphs we can express the mere existence of a clique. Only with an ordering on the vertices we can actually express that such a clique is large. In this case we shall talk about second order limit.

\section{The first order wide limit}



From now on we closely follow Chapter 1 of \cite{krajicek2010forcing}. Let $\mathcal{M}$ be the $\aleph_1$-saturated model of true arithmetic discussed in the previous chapter and let $\mathcal{G}_k$ be a wide sequence of graphs and $\Omega:=\mathcal{G}_n$ for $n\in\mathcal{M}\setminus \mathbb{N}$. One can check that graphs in $\Omega$ are all pseudofinite. The model $\M$ treats all its elements (including those which represent sets) as ``finite objects'' which lets us define uniform probability even on sets which are infinite from the set-theoretical perspective. Unlike in Krajíček's book, we will not define a structure in the arithmetical language $L_{all}$ because for us the families of functions $F$ will have their range restricted to the vertex set $\set{g_n}$ of the pseudofinite member of $\G_n$ and functions from $L_{all}$ could easily generate functions with range outside of this vertex set.

However nothing forbids us to interpret the relations and constants in $L_{all}$ so we define the language $L_{rel}$ to consist precisely of the relational and constant symbols in $L_{all}$

\begin{defi}
Let $\mathcal{A}:=\{A\in\mathcal{M};A\subseteq \Omega\}$ be the set of all subsets of $\Omega$ represented by an element in $\M$.

We define the \textbf{counting measure} as the uniform probability of $A$ when we sample $\Omega$ uniformly, so we have
\[A\in\A \to \abs{A}/\abs{\Omega},\]
the counting measure takes values in $\M$-rationals.
\end{defi}

One can check that $\A$ is a Boolean algebra, but not a $\sigma$-algebra as it is not closed under all countable unions. Indeed all singleton sets are part of $\A$ but the set of all elements with standardly many predecessors in $\Omega$ is not in $\A$.

\begin{defi}\label{defiinfinitesimal}
We call an $\M$-rational \textbf{infinitesimal} if it is smaller than all standard fractions $\frac{1}{k},k\in\N$.

Define an ideal in $\A$ as $\I:=\{A\in\A;\abs{A}/\abs{\Omega}\text{ is infinitesimal}\}$. Define the Boolean algebra $\B:=\A/\I$. The induced measure on $\B$ is a real-valued measure and can be written as \[\mu(A/\I)=\st(\abs{A}/\abs{\Omega}).\]
\end{defi}

We can also check, that now $\mu$ is a measure in the ordinary sense and that $\B$ is an $\sigma$-algebra. In fact, the following key lemma holds.

\begin{lemm}
$\B$ is a complete Boolean algebra.
\end{lemm}

The maximal and minimal element in $\B$ will be denoted $\1$ and $\0$ respectively. We now define what we require of the family of functions $F$ we already mentioned.

\begin{defi}
Let $F$ be a non-empty set of $\M$-finite function which are elements in $\M$. We call it a \textbf{(random) vertex family} if it satisfies the following:
\begin{itemize}
\item The domain of any function $\alpha\in F$ is $\Omega$ and the range is $\set{g_n}$.
\end{itemize}
\end{defi}

Note that while every $\alpha\in F$ is represented by some element in $\M$, this need not be the case for the whole family $F$. 

%From now on $E$ denotes a binary relational symbol which we understand as `limit edge relation'. For a $\{E\}$-formula $\varphi$ we denote its translation to a $\{E\}$-formula by rewriting every $E$ to $E$ as $\varphi^{E}$. 

%( (!!!) ? Neměl bych mluvit v celé práci jen o jazyce $\{E\}$? Jméno pro limitní objekt stejně máme jiné.)

Now we can finally define the first order wide limit.

\begin{defi}
We define a $\B$-valued $\{E\}$-structure $\lim_{k\to n}^{F}G_k$, with universe $F$ and $\{E\}$-sentences being evaluated by the following inductive conditions:
\begin{itemize}
\item $\bbl \alpha=\beta \bbr := \{\omega\in\Omega; \alpha(\omega)=\beta(\omega)\}/\I.$
\item $\bbl E(\alpha,\beta) \bbr := \{\omega\in\Omega; E_\omega(\alpha,\beta))\}/\I$.
\item $\bbl-\bbr$ commutes with $\land$,$\lor$,$\lnot$.
\item $\bbl(\exists x)A(x)\bbr:=\Lor_{\alpha\in F}\bbl A(\alpha)\bbr$.
\item $\bbl(\forall x)A(x)\bbr:=\Land_{\alpha\in F}\bbl A(\alpha)\bbr$.
\end{itemize}

By abuse of notation we will usually denote the limit $\lim_{F}\G_n$.
\end{defi}

We will also define the structure $K(\G_n,F)$ which is not important in the first order case, but makes the definition of its second order counterpart more manageable. This structure corresponds to a fragment, a substructure of a $L_{rel}$-reduct to be precise, of structures $K(F)$ defined in \cite{krajicek2010forcing} for some larger family $F$.


\begin{defi}
$K(\G_n,F)$ will denote a $\B$-valued $L_{rel}\cup\{E\}$-structure defined as an $L_{rel}\cup\{E\}$-expansion of $\lim_F\G_n$. The Boolean evaluations of atomic $L_{rel}$-sentences are defined as follows:

\begin{itemize}
\item $\bbl R(\alpha_1,\dots,\alpha_k) \bbr := \{\omega\in\Omega; R(\alpha_1,\dots,\alpha_k)\}/\I$ for any $k$-ary $R\in L_{rel}$.
\end{itemize}

We call \textbf{the arithmetical expansion of the (first order) wide limit}.
\end{defi}


\section{The second order wide limit}

While we can find a truth value of a sentence in the language of graphs in the limit $\lim_{F}\G_n$, we will encounter situations where this is not sufficient to analyze the wide sequence $\{\G_k\}_{k>0}$. 

In Chapter \ref{chapdense} we will investigate how the existence of large cliques corresponds to the size of cliques in the limit graph. First we need some way to witness subsets of vertices -- this leads us to the second order wide limit. However, in the second order case the arithmetical expansion is much more important because we cannot just measure the set-theoretical cardinality of any such clique. For specific $n$ we could very well have $\card(\set{\floor{\log{n}}})=\card(\set{\floor{\frac{n}{2}}})$ but from the point of view of complexity theory, cliques of size $\floor{\log n}$ and $\floor{\frac{n}{2}}$ are dramatically different. In other words, our goal is also to have means to count the number elements of subsets or relations with values in (random variables in) $\M$.

When we say second order we mean two-sorted first order structures where one sort represents the usual elements (the `first order' sort) and the other represents functions on those elements (the `second order' sort). The `second order' here can also represent sets and relations by \{0,1\} valued functions.

\begin{defi}
We call a set of functions $G\subseteq \M$ an $F$-compatible \textbf{functional family} if every $\Theta\in G$ assigns to every $\omega\in\Omega$ a function $\Theta_\omega\in\M$ and after we define
\[\Theta(\alpha)(\omega):=
\begin{cases}\Theta_\omega(\alpha(\omega))&\alpha(\omega)\in\dom(\Theta_\omega)\\0&\text{otherwise,}
\end{cases}\]
we have that for every $\alpha \in F$ and $\Theta \in G$ we have $\Theta(\alpha)\in F$.
\end{defi}

\begin{defi}
Let $F$ be a vertex family and $G$ be an $F$-compatible functional family. We define the structure $K(\G_n,F,G)$ as a two sorted $\{E\}\cup L_{rel}$-structure with sorts $F$ and $G$ interpreting the first order $L_{rel}$-sentences as in $K(\G_n,F)$ and treating the sort $G$ as follows. Variables for $G$ are treated as function symbols and can form terms with variables for $F$. For equality we let
\[\bbl\Theta=\Xi \bbr := \{\omega\in \Omega;\Theta_\omega=\Xi_\omega\}/\I\]
and for the second order quantifiers we have the following inductive clauses
\begin{itemize}
\item $\bbl(\exists X)A(X)\bbr:=\Lor_{\Theta\in G}\bbl A(\Theta)\bbr$
\item $\bbl(\forall X)A(X)\bbr:=\Land_{\Theta\in G}\bbl A(\Theta)\bbr.$
\end{itemize}

We define \textbf{the second order wide limit} $\lim_{F,n}^G\{\G_k\}_{k>0}$ as $\{E\}$-reduct of $K(\G_n,F,G)$, which we analogously call the \textbf{the arithmetical expansion of the (second order) wide limit}. By abuse of notation we will mostly denote the wide limit $\lim_F^G\G_n$.
\end{defi}

Let us note that if we have multiple Boolean valued structures $\S_1,\S_2,\dots$ we may add the name of the structure as a prefix to the evaluation function to get $\S_1\bbl-\bbr$ and $\S_2\bbl-\bbr$ to avoid ambiguity or to emphasize the structure where the evaluation takes place. This is different from the standard notation, which would be including the structure name in the superscript as $\bbl\dots\bbr^{\S_1}$, but it is in our case preferable for typographical reasons.
 
\section{The vertex family $F_{rud}$ and $G_{rud}$}\label{secFrud}

Throughout this thesis we will mostly work with the vertex family $F_{rud}$ which ties the properties of $\lim_F \G_n$ with decision tree complexity -- but decision trees can also be seen as an abstraction of oracle machine time we mentioned in the Preliminaries chapter.

After we choose the sequence $\{\G_k\}_{k>0}$ and $n>\N$ we again put $\Omega:=\G_n$ and define $F_{rud}$ as follows.

\begin{defi}
We define a \textbf{decision tree} to be a binary tree $T\in\M$ with a labeling of vertices and edges $\ell$. The non-leaf vertices are labeled by pairs of numbers $(u,v)$, where $u,v\in\set{g_n}$ and each edge is labeled either by $1$ or $0$. Each leaf vertex is then labeled by some element of $\set{g_n}$.

Every sample $\omega\in\Omega$ uniquely determines a path in $(T,\ell)$ by interpreting the vertex labels as ``is $(u,v)\in E_\omega$?'' and the edge labels as true $(1)$ and false $(0)$ and the path then uniquely determines an output.

We define $\T_{rud}$ to be the set of all $(T,\ell)$ of depth at most $g_n^{1/t}$ and $F_{rud}$ to be the set of all function computed by some $(T,\ell)\in \T_{rud}$. For brevity we will leave out the labeling of the trees out of the notation so a tree in $\T_{rud}$ can be denoted just by $T$.
\end{defi}

Note that if we are given a graph $G\in G_k$ on $g_k$ vertices, we need a polynomial sized circuit with $2l$ inputs, $l:=\log g_k$, to represent its edge relation. In the pseudofinite case, where $k=n$, if we are restricted to inspect/query at most $g_n^{1/t}$ edges then it corresponds to $2^{l/t}$ many queries or the subexponential oracle time.

The definition of $G_{rud}$ is a bit more involved. The functionals in it will be computed by tuples of elements from $F_{rud}$ in the following sense.

\begin{defi}
Let $\hat\beta=(\beta_0,\dots,\beta_{m-1})\in\M$ be a $m$-tuple of elements in $F_{rud}$, for any $\alpha\in F_{rud}$ and $\omega\in\Omega$ we define
\[\hat\beta(\omega)=
\begin{cases}
\beta_{\alpha(\omega)}(\omega)&\alpha(\omega)<m\\
0&\text{otherwise.}
\end{cases}\]
\end{defi}

\begin{defi}
The family $G_{rud}$ consists of all functionals $\Theta$ such that there is $m\in\M$ and some $\hat \beta=(\beta_0,\dots,\beta_{m-1})\in\M$ that computes it.
\end{defi}

\begin{lemm}
$G_{rud}$ is $(F_{rud})$-compatible.
\end{lemm}
\begin{proof}
By induction in $\M$ we have that all the depth of all the trees is bounded by $g_n^{1/t}$ for some $t>\N$.

If we take some $\Theta\in G_{rud}$ and $\alpha\in F_{rud}$ we can compute $\Theta(\alpha)$ also by a tree in $\T_{rud}$ by concatenating the trees computing $\alpha$ and $\beta_i$s.
\end{proof}

\section{Different choices of $n$}

Even though we generally pose no requirements on $n>\N$ there are examples of wide sequences for which the limit is sensitive on the choice of the non-standard number $n$. 

\begin{exam}
Let 

\[\
G_{k}:=
\begin{cases}
\{(\set{k},E);\abs{E}=2,E(0,1)\}&\text{$k$ even}\\
\{(\set{k},E);\abs{E}=1,\lnot E(0,1)\}&\text{$k$ odd.}\\
\end{cases}\]

Let $n>\N$ then
\begin{align}
\lim_{F_{rud}}\G_{2n+1} \bbl E(0,1)\bbr =\0,
\end{align}
but
\begin{align}
\lim_{F_{rud}}\G_{2n}\bbl E(0,1)\bbr=\1.
\end{align}

Even though the concrete wide limits we will investigate in the following chapters do not depend on the specific $n$, it is important to note that we cannot generally remove the parameter $n$ from the definition of the limit.
\end{exam}

\section{Theories of wide limits}

If $\lim_F\G_n$ is the first order wide limit we will be interested in which exact $\{E\}$-sentences are valid in it. By \textbf{valid} we mean that their $\bbl\dots\bbr$ value is $\1$.

\begin{defi}
We define $\Th(\lim_F\G_n)$ as the set of all valid $E$-sentences in $\lim_{F}\G_n$.
\end{defi}

In the next chapter (Theorem \ref{thrmuni}) we will see that if a universal $\{E\}$-sentence $\varphi$ holds for all $G\in\G_k$ for $k$ big enough then $\lim_F\G_n\bbl\varphi\bbr=\1$. In particular, a wide limit of undirected graphs is a Boolean valued undirected graph and a wide limit of directed graphs is a Boolean valued directed graph.

Lastly, let us recall the concept of 0-1 laws which say that a certain probability tends either to $0$ or $1$ and not to any intermediate value. Here, instead of probability, we can think about the Boolean values $\0$ and $\1$ and ask when does it happen that $\lim_F\G_n\bbl\varphi\bbr\in\{\0,\1\}$ for every $\{E\}$-sentence $\varphi$. This is exactly equivalent to the situation where the theory $\Th(\lim_F \G_n)$ is complete. Later in the thesis we prove such 0-1 law for several wide limits.
