\chapter{Sparse case and $\TFNP$}

\section{$\mathcal{G}_k=*\text{PATH}_k$}

Now we turn our attention to a wide sequence which is made up of `hardest instances in $\LEAF$'. That is, if we are given a degree 1 vertex labeled $0$ and search for another degree 1 vertex it is the hardest if there are only two degree 1 vertices and the path from $0$ to the solution is as long as possible.

\begin{defi}
We define $\pPATH_k$ (the pointed paths on $k$ vertices) as the set of all (undirected) graphs $G$ on the vertex set $[k]$, where $G$ is isomorphic to the path on $n$ vertices and $\deg_G(0)=1$.
\end{defi}

One can also see that it is not fruitful to analyze the $F_{rud}$ limit, because there are only $k-1$ edges in each $G\in\pPATH_k$ and so by Theorem \ref{thrmsparse} we have that $\lim_{F_{rud}}\pPATH_k\bbl(\exists x)(\exists y)\Gamma(x,y)\bbr=\0.$ Moreover in the type 2 version of the problem the graph is presented by an oracle which gives us the neighbour set for each vertex, so we define a new family $F$ as follows.

\begin{defi}
After we fix $n$, we define $F_{nbtree}$ as the set of all functions computed by some some labeled tree with the following shape:

\begin{itemize}
\item Each non-leaf node is labeled by some $v\in[n]$. 
\item For each $\{u,v\} \subseteq [n]$ and a node $N$ there is an outgoing edge from $N$ labeled $A$.
\item Each leaf is labeled by some $m\in \mathcal{M}_n$.
\item The depth of the tree is at most $n^{1/t}$ for some $t>\mathbb{N}$.
\end{itemize}

Computation of such a tree on a undirected graph $G$ goes as follows. We interpret the non-leaf nodes as questions "what is the neighbour set of $v$?" and the edges as answers from our graph $G$, and thus we follow a path determined by $G$ until we find a vertex for which the answer is not an edge (in which case the computation returns $0$) or until we find a leaf, in which case the computation returns the label of the leaf.
\end{defi}

We now shift out focus to analysing the ability of trees from $F_{nbtree}$ to find the non-zero degree 1 vertex in $G\in \pPATH_n$. We say a tree $T\in F_{nbtree}$ fails at a graph $G$ if $T(G)$ is not a non-zero vertex of degree one in $G$.

\begin{defi}
Let $m\leq n$ and $v\in [w]$ and $U\subseteq [w]$ with $\abs{U}\leq 2$, then we define 

\[\mathcal{G}_m^{v?=U}:=\{G\in\mathcal{G}_m;N_G(v)=U\},\]
where $N_G$ is the neighbour-set function of $G$.
\end{defi}
\begin{lemm}
There are bijections for all nonstandard $m\leq n$ and distinct $u,v,w\in [m]\setminus\{0\}$:

\begin{align}
\mathcal{G}_m^{v?=\{u,w\}}&\cong\mathcal{G}_{m-2}\times [2] \label{pPathfirstbij}\\
\mathcal{G}_m^{v?=\{u,0\}}&\cong\mathcal{G}_{m-2}\label{pPathsecbij}\\
\mathcal{G}_m^{0?=\{u\}}&\cong\mathcal{G}_{m-1}\label{pPaththrdbij}.
\end{align}
\end{lemm}
\begin{proof} (sketch)
For \eqref{pPathfirstbij} we can just contract all of {u,v,w} into one vertex and relabel the rest of the graph, leaving the orientation as a one remaining bit of information. This is obviously reversible and a bijection.

For \eqref{pPathsecbij} we can do the same, but the orientation is given by $0$.
\end{proof}

\begin{lemm}\label{lemmrelabeltree}
Let $T\in F_{nbtree}$, with root labeled $v\in[m]\setminus{0}$, we have for each $T_{v?=\{u,w\}}$ a tree $\tilde T_{v?=\{u,w\}}$ of the same depth, such that
\begin{align}
P_m(T_{v?=\{u,w\}}\text{ fails}|v?=\{u,w\})=P_{m-2}(\tilde T_{v?=\{u,w\}}).
\end{align}
For a tree $T$ with the root labeled $0$, we have a tree $\tilde T_{v?=\{u,w\}}$ of the same depth, such that
\begin{align}
P_m(T_{v?=\{u\}}\text{ fails}|v?=\{u\})=P_{m-1}(\tilde T_{v?=\{u\}}).
\end{align}
\end{lemm}
\begin{proof}(sketch) To construct the tree, we just replace all vertices in labels of $T_{v?=\{u,w\}}$ by there renumbering from the bijection in \eqref{pPathfirstbij}.

(TODO: Elaborate) One can then check that the trees $T_{v?=\{u,w\}}$ and $\tilde T_{v?=\{u,w\}}$ are isomorphic in a sense that their computation of a graph $G$ and $\tilde G$ respectively, $\tilde G$ being the corresponding $(m-2)$-vertex graph, agree with the structure of the path and that correctness of leaves is preserved under the renumbering. Essentially they emulate the same computation but on a smaller graph.
\end{proof}


\begin{lemm}
For all nonstandard $t>\mathbb{N},m\geq n-2n^{1/t}$ and $k\in [n^{1/t}+1]$ for all trees $T\in F_{nbtree}$ of depth $k$ we have
\[
P_m(\text{$T$ fails})\geq \prod_{i=0}^k\left (1-\frac{2}{m-2i-2}\right ).
\]
\end{lemm}
\begin{proof}
We proceed by induction on $k$. 

$k=0:$ We have that the probability of success of a straight guess is at most $\frac{1}{m-1}$. Therefore

\begin{align}
P(\text{$T$ fails})\geq \left (1-\frac{1}{m-1}\right ) \geq \left(1-\frac{2}{m-2}\right).
\end{align}

$(k-1) \Rightarrow k:$ First we assume that the root is labeled $0$. Then we have 

\begin{align}
P(\text{$T$ fails})&=\sum_{u\in V\setminus\{0\}}P_{m-1}(0Eu)P_{m-1}(T_{0?=\{u\}}\text{ fails}|0Eu)\\
&\geq P_{m-1}(T_{0?=\{u\}}\text{ fails}|0Eu)\\
&= P_{m-1}(\tilde T_{0?=\{u\}}\text{ fails})\\
&\geq \prod_{i=0}^{k-1}\left(1-\frac{2}{m-2i-2}\right)\\
&\geq \prod_{i=0}^{k}\left(1-\frac{2}{m-2i-2}\right).
\end{align}

Now we assume that the root is labeled $v\not= 0$. First we notice that

\begin{align}
P_m(vE0)&=\frac{1}{m-1}\\
P_m(N(V)=1)&=\frac{1}{m-1}\\
P_m(\abs{N(V)\setminus\{0\}}=2)&=1-\frac{2}{m-1},\label{pmmiddle}
\end{align}
the first two probabilities are obviously $\frac{1}{m-1}$ because they correspond to $v$ being positioned on one of the ends of the non-zero segment which has length $m-1$. The event in \eqref{pmmiddle} is the complement of the union of the first two events, which have empty intersetion, giving us that stated probability.

Then we have
\begin{align}
P_m(\text{$T$ fails})&=P_m(vE0)P_m(\text{$T$ fails}|vE0)\\
&\:\phantom{=}\:+P_m(\abs{N(v)\setminus\{0\}}=2)P_m(\text{$T$ fails}|\abs{N(v)\setminus\{0\}}=2)\\
&\:\phantom{=}\:+P_m(\abs{N(v)}=1)P_m(vE0)P_m(\text{$T$ fails}|\abs{N(v)}=1)\\
&\geq P_m(\abs{N(v)\setminus\{0\}}=2)P_m(\text{$T$ fails}|\abs{N(v)\setminus\{0\}}=2)\\
&\geq (1-\frac{2}{m-1})\\
&\:\phantom{=}\:\cdot\sum_{\substack{u,w\in V\setminus \{0\} \\ u\not= w}}P_m(v?=\{u,w\})P_m(T_{v?=\{u,w\}}\text{ fails}|v?=\{u,w\})\\
&\geq(1-\frac{2}{m-1})P_m(T_{v?=\{u_0,w_0\}}\text{ fails}|v?=\{u_0,w_0\})\label{pmfailchoice}\\
&\geq(1-\frac{2}{m-1})P_{m-2}(\tilde T_{v?=\{u_0,w_0\}}\text{ fails})\label{pmfaillemma}\\
&\geq(1-\frac{2}{m-1})\prod_{i=0}^{k-1}(1-\frac{2}{m-2i-4})\label{pmfailih}\\
&\geq(1-\frac{2}{m-2})\prod_{i=1}^{k}(1-\frac{2}{m-2i-2})\\
&\geq\prod_{i=0}^{k}(1-\frac{2}{m-2i-2}).
\end{align}
where in \eqref{pmfailchoice} we choose $u_0,w_0$ with the lowest value of \[P_m(\text{$T_{v?=\{u_0,w_0\}}|v?=\{u_0,w_0\}$}),\] the bound follows the fact that all $P_m(v?=\{u,w\})$ are the same for distinct non-zero $u,w$. In \eqref{pmfaillemma} we use the lemma \ref{lemmrelabeltree} and in \eqref{pmfailih} we use the induction hypothesis.
\end{proof}

\begin{crll}\label{crllpPathtreefail}
For a tree $T\in F_{nbtree}$ we have that
\[P_n(T\text{ fails})\approx 1.\]
\end{crll}
\begin{proof}
Since $T$ has depth at most $n^{1/t}$ for some $t>\mathbb{N}$ we by the previous lemma that
\begin{align}
P_n(\text{$T$ fails})&\geq \prod_{i=0}^{n^{1/t}}\left(1-\frac{2}{n-2i-2}\right)\\
&\geq \left(1-\frac{2n^{1/t}}{n-2n^{1/t}-2}\right)\\
&\approx 1.
\end{align}
\end{proof}

Finally we can prove the following theorem.

\begin{thrm}
\[\bbl(\exists v)(\exists u)(\forall w)((v\not=0)\land (\Gamma(v,u)) \land(\Gamma(v,w) \to u=w))\bbr=\0\]
\end{thrm}
\begin{proof}
Expanding the value of the formula in the statement we get
\[\Lor_{\alpha}\Lor_{\beta}\Land_{\gamma}\bbl(\alpha\not=0)\land (\Gamma(\alpha,\beta)) \land(\Gamma(\alpha,\gamma) \to \beta=\gamma)\bbr,\]
to prove it evalues to $\0$ we need to find for every $\alpha,\beta$ some $\gamma$ such that
\[\bbl(\alpha\not=0)\land (\Gamma(\alpha,\beta)) \land(\Gamma(\alpha,\gamma) \to \beta=\gamma)\bbr=\0.\]
For any $\alpha,\beta$ we define
\[\gamma(\omega):=\begin{cases}v&N(\alpha(\omega))=\{v\}\\w&w\in N(\alpha(\omega))\setminus\{\beta(\omega)\},\end{cases}\]
such a function can be computed by a tree in $F_{nbtree}$ which we can construct by concatenation of trees computing $\alpha$ and $\beta$.

Let $T$ be the tree computing $\alpha$. Now we proceed by contradiction, let
\[\epsilon:=\mu(\bbl(\alpha\not=0)\land (\Gamma(\alpha,\beta)) \land(\Gamma(\alpha,\gamma) \to \beta=\gamma)\bbr)>0,\]
by definition this means that
\[\epsilon=\st(P_n[(\alpha\not=0)\land (\Gamma(\alpha,\beta)) \land(\Gamma(\alpha,\gamma) \to \beta=\gamma)])>0.\]
Expanding the value of the formula in the statement we get
\[\Lor_{\alpha}\Lor_{\beta}\Land_{\gamma}\bbl(\alpha\not=0)\land (\Gamma(\alpha,\beta)) \land(\Gamma(\alpha,\gamma) \to \beta=\gamma)\bbr,\]
to prove it evalues to $\0$ we need to find for every $\alpha,\beta$ some $\gamma$ such that
\[\bbl(\alpha\not=0)\land (\Gamma(\alpha,\beta)) \land(\Gamma(\alpha,\gamma) \to \beta=\gamma)\bbr=\0.\]
For any $\alpha,\beta$ we define
\[\gamma(\omega):=\begin{cases}v&N(\alpha(\omega))=\{v\}\\w&w\in N(\alpha(\omega))\setminus\{\beta(\omega)\},\end{cases}\]
such a function can be computed by a tree in $F_{nbtree}$ which we can construct by concatenation of trees computing $\alpha$ and $\beta$.

Let $T$ be the tree computing $\alpha$. Now we proceed by contradiction, let
\[\epsilon:=\mu(\bbl(\alpha\not=0)\land (\Gamma(\alpha,\beta)) \land(\Gamma(\alpha,\gamma) \to \beta=\gamma)\bbr)>0,\]
by definition this means that
\[\epsilon=\st(P_n[(\alpha\not=0)\land (\Gamma(\alpha,\beta)) \land(\Gamma(\alpha,\gamma) \to \beta=\gamma)])>0.\]
But by definition of $\gamma$ and Corollary \ref{crllpPathtreefail} we have
\begin{align*}
0&<\epsilon\\
&=\st(P_n[(\alpha\not=0)\land (\Gamma(\alpha,\beta)) \land(\Gamma(\alpha,\gamma) \to \beta=\gamma)])\\
&\leq\st(P_n[(\alpha\not=0)\land (\Gamma(\alpha,\beta)) \land \abs{N(\alpha)}=1])\\
&\leq\st(P_n[(\alpha\not=0)\land \abs{N(\alpha)}=1])\\
&=\st(P_n[T\text{ does not fail}])\\
&=0.
\end{align*}
A contradiction.
\end{proof}

(TODO: General method to compute these, directed path, path up to $n$, conjecture about $K(F,G)$)
