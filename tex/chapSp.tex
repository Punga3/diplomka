\chapter{Sparse case and \texorpdfstring{$\TFNP$}{TFNP}}\label{chapsparse}

\section{\texorpdfstring{$\mathcal{G}_k=*\text{PATH}_k$}{Gk=*PATHk}}

Now we turn our attention to a wide sequence which is made up of `the hardest instances in $\LEAF$'. That is, if we are given a degree 1 vertex labeled $0$ and search for another degree 1 vertex, it is the hardest if there are only two degree 1 vertices and the path from $0$ to the solution is as long as possible.

\begin{defi}
We define $\pPATH_k$ (the pointed paths on $k$ vertices) as the set of all (undirected) graphs $G$ on the vertex set $\set{k}$, where $G$ is isomorphic to the path on $k$ vertices and $\deg_G(0)=1$.
\end{defi}

One can also see that it is not fruitful to analyze the $F_{rud}$ limit, because there are only $k-1$ edges in each $G\in\pPATH_k$ and therefore, by an analogous argument to the proof of Theorem \ref{thrmsparse}, we have that $\lim_{F_{rud}}\pPATH_k\bbl(\exists x)(\exists y)E(x,y)\bbr=\0.$ Moreover in the type 2 version of the problem the graph is presented by an oracle which gives us the neighbour set for each vertex, so we define a new family $F$ as follows.

\begin{defi}
After we fix $n$, we define $F_{nbtree}$ as the set of all functions computed by some some labeled tree with the following shape:

\begin{itemize}
\item Each non-leaf node is labeled by some $w\in\set{n}$. 
\item For each $\{u,v\} \subseteq \set{n}$ and a node $w$ there is an outgoing edge from $w$ labeled $\{u,v\}$.
\item Each leaf is labeled by some $m\in \set{g_n}$.
\item The depth of the tree is at most $g_n^{1/t}$ for some $t>\mathbb{N}$.
\end{itemize}

Computation of such a tree on an undirected graph $G$ goes as follows. We interpret the non-leaf nodes as questions "what is the neighbour set of $w$?" and the edges as answers from our graph $\omega$, and thus we follow a path in the computation tree determined by $\omega$ until we find a leaf, in which case the computation returns the label of the leaf. We will denote the set of all such trees as $\T_{nbtree}$.
\end{defi}

We now shift our focus to analyzing the ability of trees from $F_{nbtree}$ to find the non-zero degree 1 vertex in $G\in \pPATH_n$. We say that a tree $T\in \T_{nbtree}$ fails at a graph $G$ if $T(G)$ is not a non-zero vertex of degree one in $G$.

\begin{defi}
Let $m\leq n$ and $v\in \set{w}$ and $U\subseteq \set{w}$ with $\abs{U}\leq 2$, then we define 

\[\mathcal{G}_m^{v?=U}:=\{G\in\mathcal{G}_m;N_G(v)=U\},\]
where $N_G$ is the neighbour-set function of $G$.
\end{defi}
\begin{lemm}
There are bijections for all nonstandard $m\leq n$ and distinct $u,v,w\in \set{m}\setminus\{0\}$:

\begin{align}
\mathcal{G}_m^{v?=\{u,w\}}&\cong\mathcal{G}_{m-2}\times [2] \label{pPathfirstbij}\\
\mathcal{G}_m^{v?=\{u,0\}}&\cong\mathcal{G}_{m-2}\label{pPathsecbij}\\
\mathcal{G}_m^{0?=\{u\}}&\cong\mathcal{G}_{m-1}\label{pPaththrdbij}.
\end{align}
\end{lemm}
\begin{proof} (sketch)
For \eqref{pPathfirstbij} we can just contract all of {u,v,w} into one vertex and relabel the rest of the graph, leaving the orientation as one remaining bit of information. This is obviously reversible and a bijection.

For \eqref{pPathsecbij} we can do the same, but the orientation is given by $0$.
\end{proof}

\begin{lemm}\label{lemmrelabeltree}
Let $T\in \T_{nbtree}$, with root labeled $v\in[m]\setminus{0}$, we have for each $T_{v?=\{u,w\}}$ a tree $\tilde T_{v?=\{u,w\}}$ of the same depth, such that
\begin{align}
\Pr_{G\in\G_m}(T_{v?=\{u,w\}}\text{ fails}|v?=\{u,w\})=\Pr_{G\in\G_{m-2}}(\tilde T_{v?=\{u,w\}}).
\end{align}
For a tree $T$ with the root labeled $0$, we have a tree $\tilde T_{v?=\{u,w\}}$ of the same depth, such that
\begin{align}
\Pr_{\G\in\G_m}(T_{v?=\{u\}}\text{ fails}|v?=\{u\})=\Pr_{\G_{m-1}}(\tilde T_{v?=\{u\}}).
\end{align}
\end{lemm}
\begin{proof}(sketch) To construct the tree, we just replace all vertices in labels of $T_{v?=\{u,w\}}$ by their renumbering from the bijection in \eqref{pPathfirstbij}.

One can then check that the trees $T_{v?=\{u,w\}}$ and $\tilde T_{v?=\{u,w\}}$ are isomorphic in a sense that their computation of a graph $G$ and $\tilde G$ respectively, $\tilde G$ being the corresponding $(m-2)$-vertex graph, agree with the structure of the path and that correctness of leaves is preserved under the renumbering. Essentially they emulate the same computation but on a smaller graph.
\end{proof}


\begin{lemm}\label{lemmpPATHlowerbound}
For all nonstandard $t>\mathbb{N},m\geq n-2n^{1/t}$ and $k\in [n^{1/t}+1]$ for all trees $\T\in \T_{nbtree}$ of depth $k$ we have
\[
\Pr_{G\in\G_m}(\text{$T$ fails})\geq \prod_{i=0}^k\left (1-\frac{2}{m-2i-2}\right ).
\]
\end{lemm}
\begin{proof}
We proceed by induction on $k$. 

$k=0:$ We have that the probability of success of a straight guess is at most $\frac{1}{m-1}$. Therefore

\begin{align}
\Pr_{G\in\G_m}(\text{$T$ fails})\geq \left (1-\frac{1}{m-1}\right ) \geq \left(1-\frac{2}{m-2}\right).
\end{align}

$(k-1) \Rightarrow k:$ First we assume that the root is labeled $0$. Then we have 

\begin{align}
\Pr_{G\in\G_m}[\text{$T$ fails}]&=\sum_{u\in V\setminus\{0\}}\Pr_{G\in\G_{m-1}}[E(0,u)]\Pr_{G\in\G_{m-1}}[T_{0?=\{u\}}\text{ fails}|E(0,u)]\\
&\geq \Pr_{G\in\G_{m-1}}[T_{0?=\{u\}}\text{ fails}|E(0,u)]\\
&= \Pr_{G\in\G_{m-1}}[\tilde T_{0?=\{u\}}\text{ fails}]\\
&\geq \prod_{i=0}^{k-1}\left(1-\frac{2}{m-2i-2}\right)\\
&\geq \prod_{i=0}^{k}\left(1-\frac{2}{m-2i-2}\right).
\end{align}

Now we assume that the root is labeled $v\not= 0$. First we notice that

\begin{align}
\Pr_{G\in\G_m}[E(v,0)]&=\frac{1}{m-1}\\
\Pr_{G\in\G_m}[N(V)=1]&=\frac{1}{m-1}\\
\Pr_{G\in\G_m}[\abs{N(V)\setminus\{0\}}=2]&=1-\frac{2}{m-1},\label{pmmiddle}
\end{align}
the first two probabilities are obviously $\frac{1}{m-1}$ because they correspond to $v$ being positioned on one of the ends of the non-zero segment which has length $m-1$. The event in \eqref{pmmiddle} is the complement of the union of the first two events, which have empty intersection, giving us that stated probability.

Then we have for $p:=\Pr_{G\in\G_m}[\text{$T$ fails}]$
\begin{align}
p&=\Pr_{G\in\G_m}[E(v,0)]\Pr_{G\in\G_m}[\text{$T$ fails}|E(v,0)]\\
&\:\phantom{=}\:+\Pr_{G\in\G_m}[\abs{N(v)\setminus\{0\}}=2]\Pr_{G\in\G_m}[\text{$T$ fails}|\abs{N(v)\setminus\{0\}}=2]\\
&\:\phantom{=}\:+(\Pr_{G\in\G_m}[\abs{N(v)}=1]\Pr_{G\in\G_m}[E(v,0)]\Pr_{G\in\G_m}[\text{$T$ fails}|\abs{N(v)}=1])\\
&\geq \Pr_{G\in\G_m}[\abs{N(v)\setminus\{0\}}=2]\Pr_{G\in\G_m}[\text{$T$ fails}|\abs{N(v)\setminus\{0\}}=2]\\
&\geq (1-\frac{2}{m-1})\\
&\:\phantom{=}\:\sum_{\substack{u,w\in V\setminus \{0\} \\ u\not= w}}\Pr_{G\in\G_m}[v?=\{u,w\}]\Pr_{G\in\G_m}[T_{v?=\{u,w\}}\text{ fails}|v?=\{u,w\}]\\
&\geq(1-\frac{2}{m-1})\Pr_{G\in\G_m}[T_{v?=\{u_0,w_0\}}\text{ fails}|v?=\{u_0,w_0\}]\label{pmfailchoice}\\
&\geq(1-\frac{2}{m-1})\Pr_{G\in\G_{m-2}}[\tilde T_{v?=\{u_0,w_0\}}\text{ fails}]\label{pmfaillemma}\\
&\geq(1-\frac{2}{m-1})\prod_{i=0}^{k-1}(1-\frac{2}{m-2i-4})\label{pmfailih}\\
&\geq(1-\frac{2}{m-2})\prod_{i=1}^{k}(1-\frac{2}{m-2i-2})\\
&\geq\prod_{i=0}^{k}(1-\frac{2}{m-2i-2}).
\end{align}
where in \eqref{pmfailchoice} we choose $u_0,w_0$ with the lowest value of \[\Pr_{G\in\G_m}(\text{$T_{v?=\{u_0,w_0\}}|v?=\{u_0,w_0\}$}),\] the bound follows the fact that all $\Pr_{G\in\G_m}(v?=\{u,w\})$ are the same for distinct non-zero $u,w$. In \eqref{pmfaillemma} we use the lemma \ref{lemmrelabeltree} and in \eqref{pmfailih} we use the induction hypothesis.
\end{proof}

\begin{crll}\label{crllpPathtreefail}
For a tree $T\in \T_{nbtree}$ and $c\in \N$ we have that
\[\st\left(\Pr_{G\in\G_{n-c}}[T\text{ fails}]\right)= 1.\]
\end{crll}
\begin{proof}
Since $T$ has depth at most $n^{1/t}$ for some $t>\mathbb{N}$ we by the previous lemma that
\begin{align}
\Pr_{G\in\G_n}(\text{$T$ fails})&\geq \prod_{i=0}^{n^{1/t}}\left(1-\frac{2}{n-2i-c-2}\right)\\
&\geq \left(1-\frac{2n^{1/t}}{n-2n^{1/t}-c-2}\right)
\end{align}
and the standard part of the lower bound is $1$.
\end{proof}

Finally, we can prove the following theorem.

\begin{thrm}\label{thrmnoend}
\[\lim_{F_{nbtree}}\pPATH_n\bbl(\exists v)(\exists u)(\forall w)((v\not=0)\land (E(v,u)) \land(E(v,w) \to u=w))\bbr=\0\]
\end{thrm}
\begin{proof}
Expanding the value of the formula in the statement we get
\[\Lor_{\alpha}\Lor_{\beta}\Land_{\gamma}\bbl(\alpha\not=0)\land (E(\alpha,\beta)) \land(E(\alpha,\gamma) \to \beta=\gamma)\bbr,\]
to prove it evaluates to $\0$ we need to find for every $\alpha,\beta$ some $\gamma$ such that
\[\bbl(\alpha\not=0)\land (E(\alpha,\beta)) \land(E(\alpha,\gamma) \to \beta=\gamma)\bbr=\0.\]
For any $\alpha,\beta$ we define
\[\gamma(\omega):=\begin{cases}v&N(\alpha(\omega))=\{v\}\\w&w\in N(\alpha(\omega))\setminus\{\beta(\omega)\},\end{cases}\]
such a function can be computed by a tree in $F_{nbtree}$ which we can construct by concatenation of trees computing $\alpha$ and $\beta$.

Let $T$ be the tree computing $\alpha$. Now we proceed by contradiction, let
\[\epsilon:=\mu(\bbl(\alpha\not=0)\land (E(\alpha,\beta)) \land(E(\alpha,\gamma) \to \beta=\gamma)\bbr)>0,\]
by definition this means that
\[\epsilon=\st(P_n[(\alpha\not=0)\land (E(\alpha,\beta)) \land(E(\alpha,\gamma) \to \beta=\gamma)])>0.\]
Expanding the value of the formula in the statement we get
\[\Lor_{\alpha}\Lor_{\beta}\Land_{\gamma}\bbl(\alpha\not=0)\land (E(\alpha,\beta)) \land(E(\alpha,\gamma) \to \beta=\gamma)\bbr,\]
to prove it evalues to $\0$ we need to find for every $\alpha,\beta$ some $\gamma$ such that
\[\bbl(\alpha\not=0)\land (E(\alpha,\beta)) \land(E(\alpha,\gamma) \to \beta=\gamma)\bbr=\0.\]
For any $\alpha,\beta$ we define
\[\gamma(\omega):=\begin{cases}v&N(\alpha(\omega))=\{v\}\\w&w\in N(\alpha(\omega))\setminus\{\beta(\omega)\},\end{cases}\]
such a function can be computed by a tree in $F_{nbtree}$ which we can construct by concatenation of trees computing $\alpha$ and $\beta$.

Let $T$ be the tree computing $\alpha$. Now we proceed by contradiction, let
\[\epsilon:=\mu(\bbl(\alpha\not=0)\land (E(\alpha,\beta)) \land(E(\alpha,\gamma) \to \beta=\gamma)\bbr)>0,\]
by definition this means that
\[\epsilon=\st(P_n[(\alpha\not=0)\land (E(\alpha,\beta)) \land(E(\alpha,\gamma) \to \beta=\gamma)])>0.\]
But by definition of $\gamma$ and Corollary \ref{crllpPathtreefail} we have
\begin{align*}
0&<\epsilon\\
&=\st(P_n[(\alpha\not=0)\land (E(\alpha,\beta)) \land(E(\alpha,\gamma) \to \beta=\gamma)])\\
&\leq\st(P_n[(\alpha\not=0)\land (E(\alpha,\beta)) \land \abs{N(\alpha)}=1])\\
&\leq\st(P_n[(\alpha\not=0)\land \abs{N(\alpha)}=1])\\
&=\st(P_n[T\text{ does not fail}])\\
&=0.
\end{align*}
A contradiction.
\end{proof}

\begin{crll}\label{crllpPATH}
$\Th(\lim_{F_{nbtree}}\pPATH_n)$ is complete.
\end{crll}
\begin{proof}
By applying Theorem \ref{thrmuni} we have that the sentences $\lnot C_k$ stating that there are not cycles of length $k\in\N$ are valid all in $\lim_{F_{nbtree}}\pPATH_n$, the sentence $D^{1,2}_{1,rest}$ stating that there is exactly one vertex of degree $1$ and all other vertices have degree $2$ is valid by Theorem \ref{thrmnoend}. 

Let $T=\{\lnot C_k,k\in\N\}\cup\{D_{1,rest}^{1,2}\}$, and let $\A_1,\A_2\models T$, then we can see by the handshaking lemma that $\M_1$ and $\M_2$ are both infinite. And we can see that they can be decomposed into one path starting at $0$ with no end, and then more infinite paths which have the order type of $\Z$. The duplicator of Ehrenfeucht-Fraïssé game has a winning strategy by responding to any element on the order type $\Z$ with a far enough element on the path of the order $\N$.
\end{proof}

\section{\texorpdfstring{$\G_k=\pPATHl_k$}{Gk=*PATHlk}}

So far we have proved that the hardest instances of $\LEAF$ do not have a solution in the $F_{nbtree}$ limit and that they satisfy the 0-1 law. We can generalize this result to a larger class of instances.

\begin{defi}
We define $\pPATHl_k$ (the pointed paths on $k$ vertices of length at most $k$) as the set of all (undirected) graphs $G$ on the vertex set $\set{k}$, where $G$ has a subgraph isomorphic to the path on $l\leq k$ vertices, $\deg_G(0)=1$ and no other edges.
\end{defi}

Immediately we have that $\pPATH_k$ is a portion of $\pPATHl_k,$ we can prove even more.

\begin{defi}
We define $\pPATH_k^{\l}$ as the portion of $\pPATHl_k$ where the path subgraph is of length exactly $l$. 
\end{defi}


\begin{lemm}
Let $c\in \N$, then 
\[\lim_{k\to\infty}\frac{\abs{\pPATH_k^{k-c}}}{\abs{\pPATHl_k}}=\frac{1}{ec!}\]
So $\pPATH_k^{k-c}$ is a large portion of $\pPATHl_k$ and specially $\pPATH_k\leqp \pPATHl_k$.
\end{lemm}
\begin{proof}
By direct computation we have that the fraction $\abs{\pPATH_k}/\abs{\pPATHl_k}$ is
\begin{align}
\frac{(k-1)!/(c!)}{\sum_{i=1}^{k-1}\prod_{j=0}^{i-1}(k-j-1)}&=\frac{(k-1!)}{c!\sum_{i=1}^{k-1}\frac{(k-1)!}{(k-i-1)!}}\\
&=\frac{(k-1!)}{c!\sum_{i=1}^{k-c-1}\frac{(k-1)!}{(k-i-1)!}}\\
&=\frac{1}{c!\sum_{i=1}^{k-1}\frac{1}{(k-i-1)!}}\\
&=\frac{1}{c!\sum_{i=0}^{k-2}\frac{1}{i!}},
\end{align}
and the denominator tends to $ec!$ as $k\to\infty$.
\end{proof}

\begin{lemm}\label{lemmpathnnc}
Let $T\in \T_{nbtree}$ be a tree, then
\[\st\left(\Pr_{G\in\pPATH_{n}^{n-c}}[\text{$T$ fails}]\right )=1.\]
\end{lemm}
\begin{proof}(sketch)
By an analogous argument to how we proved Lemma \ref{lemmpPATHlowerbound} we get that
\[\Pr_{G\in\pPATH_{n}^{n-c}}[T\text{ fails}]\geq \prod_{i=0}^{n^{1/t}}\left(1-\frac{(c+2)}{n-2i-2}\right),\]
standard part of which is also $1$. The constant $c$ appears because there is a $(1-\frac{c}{m})$ chance of finding a degree $0$ in the induction step.
\end{proof}

\begin{lemm}\label{lemmpathmaj}
Let $\G_k':=\bigcup_{c\in\N}\pPATH^{k-c}_k$, then
\[\lim_{k\to\infty}\frac{\abs{\G_k'}}{\abs{\pPATHl_k}}=1.\]
In other words, $\G_k'$ is a major portion of $\pPATHl_k$.
\end{lemm}
\begin{proof}
There is a (elementwise least) increasing sequence $\{c_k\}_{k>0}$ such that for each $k_0>0$ and $k\geq k_0$ we have $\pPATH_k^{k-{c_{k_0}}}\neq \varnothing$, moreover $\lim_{k\to\infty} c_k = \infty$.

$\pPATH^{k-c}_k$ are disjoint for different choices of $c$, so by direct computation we have that the fraction $\abs{\G_k'}/\abs{\pPATHl_k}$ is
\begin{align}
\sum_{c=0}^{c_k} \frac{\abs{\pPATH_k^{k-c}}}{\abs{\pPATHl_k}}&=\sum_{c=0}^{c_k}\frac{1}{c!\sum_{i=0}^{k-2}\frac{1}{i!}}\\
&=\left(\frac{1}{\sum_{i=0}^{k-2}\frac{1}{i!}}\right)\sum_{c=0}^{c_k}\frac{1}{c!},
\end{align}
which tends to $1$ as $k\to\infty$.
\end{proof}

\begin{thrm}
\[\lim_{F_{nbtree}}\pPATHl_n\bbl(\exists v)(\exists u)(\forall w)((v\not=0)\land (E(v,u)) \land(E(v,w) \to u=w))\bbr=\0\]
\end{thrm}
\begin{proof}(sketch)
The probability that a tree fails on any of the wide sequences $\pPATH_k^{k-c}$ is infinitesimal due to Lemma \ref{lemmpathnnc}. This is true for any of their unions so for $\G_k'$ in particular. In Lemma \ref{lemmpathmaj} we proved that $\G_k'$ is a major portion of $\pPATHl_k$, so by Lemma \ref{lemmporup} the probability that a tree fails in $\pPATHl$ is arbitrarily close to $1$.
\end{proof}

So we have for a larger set of instances that their wide limit has no solution relative to $F_{nbtree}$. Moreover, we have the following.

\begin{crll}\label{crllpPATHl}
\[\Th\left(\lim_{F_{nbtree}}\pPATHl_n\right)=\Th\left(\lim_{F_{nbtree}}\pPATH_n\right)\]
\end{crll}
\begin{proof}
Since the wide sequences $\pPATH_k^{k-c}$, for $c\in\N$, partition a major portion of $\pPATHl_k$, and in each of them the probability of a tree in $F_{nbtree}$ inspecting a degree 0 vertex is infinitesimal, we have that the wide limit of the major portion $\G_k'$ satisfies the theory $\Th(\lim_{F_{nbtree}}\pPATH_n)$ and therefore so does $\lim_{F_{nbtree}}\pPATHl_n$ .
\end{proof}



\section{\texorpdfstring{$\G_k=\pDPATH_k$}{Gk=*DPATHk}}\label{secpDPATH}

As $\pPATH_k$ was the wide sequence consisting of the hardest instances of $\LEAF$ the complete problem for $\PPA$, we define $\pDPATH_k$ analogously but in the directed case so it consists of the hardest instances of $\SOURCEORSINK$ the complete problem for $\PPAD$.


\begin{defi}
We define $\pDPATH_k$ (the pointed directed paths on $k$ vertices) as the set of all directed graphs $G$ on the vertex set $\set{k}$, where $G$ is isomorphic to the path on $k$ vertices such that $\deg_G^{+}(0)=0$ and $\deg_G^{-}(1)=1$.
\end{defi}

But now, since we are working with directed graphs which have two types of neighbour sets $N_G^+(v)=\{w\in V_G;E_G(w,v)\}$ and $N_G^-(v)=\{w\in V_G;E_G(v,w)\}$, we would like to define a family $F_{dtree}$ of those trees which can inspect either of the neighbour sets.

\begin{defi}
After we fix $n$, we define $F_{dtree}$ as the set of all functions computed by some labeled tree with the following shape:

\begin{itemize}
\item Each non-leaf node is labeled by some $v\in\set{n}$ and a symbol $\circ\in\{+,-\}$. 
\item For each $v \in \set{n}$ and a node $w$ there is an outgoing edge from $w$ labeled $\{v\}$ and also an outgoing edge labeled $\varnothing$.
\item Each leaf is labeled by some $m\in \set{g_n}$.
\item The depth of the tree is at most $g_n^{1/t}$ for some $t>\mathbb{N}$.
\end{itemize}

Computation of such a tree on an undirected graph $G$ goes as follows. We interpret the non-leaf nodes as questions "what is $N_G^\circ(v)$?" and the edges as answers from our graph $G$, and thus we follow a path determined by $G$ until we find a leaf, in which case the computation returns the label of the leaf.

We denote the set of such trees $\T_{dtree}$.
\end{defi}

We will not cover details, but analysis of these trees in $\T_{dtree}$ finding the nonzero sink is more or less the same as the $F_{nbtree}$ case for $\pPATH_k$, so we have the following.

\begin{thrm}
\[\lim_{F_{nbtree}}\pDPATH_n\bbl(\exists v)(\forall w)((v\not=0)\land \lnot E(v,w))\bbr=\0\]
\end{thrm}

\begin{crll}\label{crllpDPATH}
$\Th(\lim_{F_{nbtree}}\pDPATH_n)$ is complete.
\end{crll}

In the type 2 complexity theory of $\TFNP^2$ we know that there is no (oracle polynomial time) reduction from $\LEAF$ to $\SOURCEORSINK$. An important question arises -- is this reflected in the second order arithmetical expansion $K(\pPATH_n,F_{nbtree},G_{nbtree})$? Where $G_{nbtree}$ is defined analogously as $G_{rud}$ but with the components in the tuple from $F_{nbtree}$. More specifically we ask the following.

\begin{ques}
Consider an instance of $\SOURCEORSINK$ defined by some $\Theta\in G_{nbtree}$. Let $\varphi_\Theta$ be the first order statement which says `$\Theta$ has a solution'. Are all $\varphi_\Theta$ valid in $K(\pPATH_n,F_{nbtree},G_{nbtree})$?
\end{ques}
