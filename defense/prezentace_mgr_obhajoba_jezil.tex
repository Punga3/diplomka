\documentclass{beamer}
\usepackage[utf8]{inputenc}
\usepackage[czech]{babel}
\usepackage{amssymb}
\usepackage{amsthm}
\usepackage{amsmath}
\usepackage{tikz}
\usepackage{stmaryrd}

\title{Pseudokonečné struktury a limity}
\author{Ondřej Ježil}
\date{9. června 2022}

\usetheme{Boadilla}
\newcommand{\N}{\mathbb{N}}
\renewcommand{\P}{\mathcal{P}}
\newcommand{\Q}{\mathbb{Q}}
\newcommand{\R}{\mathbb{R}}
\newcommand{\M}{\mathcal{M}}
\newcommand{\A}{\mathcal{A}}
\newcommand{\B}{\mathcal{B}}
\newcommand{\I}{\mathcal{I}}
\newcommand{\st}{\text{st}}
\newcommand{\bbl}{\llbracket}
\newcommand{\bbr}{\rrbracket}
\newcommand{\G}{\mathcal{G}}
\newcommand{\0}{\textbf{0}}
\newcommand{\1}{\textbf{1}}
\newcommand{\abs}[1]{\lvert #1 \rvert}
\newcommand{\Th}{\text{Th}}
\newcommand{\EDGE}{\text{EDGE}}
\newcommand{\ALL}{\text{ALL}}
\newcommand{\nonEDGE}{\text{nonEDGE}}
\newcommand{\SK}{\text{SK}}
\newcommand{\CK}{\text{CK}}
\newcommand{\NP}{\textbf{NP}}
\newcommand{\FP}{\textbf{FP}}
\newcommand{\pPATH}{*\text{PATH}}
\newcommand{\TFNP}{\textbf{TFNP}}
\newcommand{\PPA}{\textbf{PPA}}
\newtheorem{thrm}{Věta}
\newtheorem{ques}{Otázka}
\newtheorem{exam}{Příklad}
\newtheorem{defi}{Definice}

\begin{document}

\frame{\titlepage}
\begin{frame}
\frametitle{Úvod -- kontext}
\begin{itemize}[<+->]
\item Práce se pohybuje na rozmezí matematické logiky a teorie složitosti
\item \textbf{Limita struktur:}
posloupnost konečných grafů $\to$ nekonečný \uv{graf}, zachycuje vlastnosti dané posloupnosti
\item Mnoho konstrukcí (ultraprodukt, věta o kompaktnosti, grafony, \ldots)
\item Využití (0-1 zákony, extremální kombinatorika, \ldots)
\end{itemize}
\end{frame}

\begin{frame}
\frametitle{Cíl práce}
\begin{itemize}[<+->]
\item V práci definuji nový pojem tzv. široké limity
\item \textbf{Intuice}: třída vstupů výpočetního problému $\to$ limitní objekt, vlastnosti $\approx$ jak algoritmy dané složitosti \uv{vidí obecný vstup}
\item Klíčová metoda: Forsing s náhodnými proměnnými
\item Cíl práce:
\begin{itemize}
\item Vyvinout základní teorii
\item Předvést příklady
\item \textbf{Svázat vlastnosti těchto limit s teorií složitosti}
\end{itemize}
\item Následuje neformální popis široké limity
\end{itemize}
\end{frame}

\begin{frame}
\frametitle{Výchozí situace -- široká posloupnost}
\begin{defi}
Posloupnost konečných množin konečných (ne)orientovaných grafů $\{\G_k\}_{k>0}$ se nazývá \textbf{široká posloupnost}, pokud platí, že
\begin{itemize}
\item existuje rostoucí posloupnost $\{g_k\}_{k>0}$ tak, že pro každé $G\in\G_k$ je množina vrcholů $V_{G}=\{0,\dots,g_k-1\}$,
\item $\lim_{k\to\infty}\abs{\G_k}=\infty.$
\end{itemize}
\end{defi}
\pause
\begin{itemize}
\item $\G_k$: grafy velikosti $g_k$\pause
\item Třída grafů $\rightleftharpoons$ široká posloupnost\pause 
\item Příklady: grafy s právě jednou hranou, grafy s přesně jednou velkou klikou, grafy obsahující trojúhelník, \ldots
\end{itemize}
\end{frame}

\begin{frame}
\frametitle{Nestandardní model(y) aritmetiky}
\begin{itemize}[<+->]
\item Matematická logika: $\exists \M \models \Th(\N)$, $\M\not\cong\N$
\item Fixujeme nestandardní model $\M$ (splňující tech. podmínkou)
\item $\M$ je tedy polookruh, vlastnosti prvního řádu se shodují s $\N$ (axiomy polookruhů, indukce, Velká Fermatova věta, \ldots)
\item Prvky $\M$ můžeme: sčítat, násobit, $\lfloor \frac{x}{2}\rfloor$, $\lfloor\log(x)\rfloor$, \ldots
\item Dosadit jako index v široké posloupnosti: $\G_n$
\item Získáme nekonečnou množinu pseudokončených grafů
\end{itemize}
\end{frame}

\begin{frame}
\frametitle{Idea definice široké limity}
\begin{itemize}[<+->]
\item Třída funkcí $F$: vstupy jsou grafy, výstupy jsou vrcholy
\item $F$ -- náhodné vrcholy v $\{0,\dots,g_n-1\}$ 
\item Na $F$ definujeme (Booleovsky ohodnocený) graf $\lim_F \G_n$.
\item Pro $\alpha,\beta\in F$ máme pravdivostní hodnotu $\bbl E(\alpha,\beta) \bbr$.
\item Tato hodnota může být buď $\1$ (pravda), $\0$ (nepravda) nebo někde \uv{mezi}.
\item Pravdivé sentence v $\lim_F\G_n \rightleftharpoons$ vlastnosti které umí $F$ dosvědčit
\end{itemize}
\end{frame}

\begin{frame}
\frametitle{Třída funkcí $F_{rud}$}
\begin{defi}
$F_{rud}$ sestává ze všech náhodných vrcholů počítáných rozhodovacími stromy hloubky nejvýše $n^{1/t}$, pro nějaké nestandardní $t$, následujícího tvaru.
\begin{center}
\begin{tikzpicture}[%
level 1/.style={sibling distance=5cm},
level 2/.style={sibling distance=3cm},
every node/.style = {draw, minimum width=1.5cm, minimum height=.75cm, anchor=north},
edge from parent path={(\tikzparentnode.south) -- (\tikzchildnode.north)}]
]
\node[] {$E(1,2)?$}
child { node[] {$E(1,3)?$}
    child { node[] {$1$} 
        edge from parent node[left,draw=none] {1}
	}
    child { node[] {$5$} 
        edge from parent node[right,draw=none] {0}
	}
    edge from parent node[left,draw=none] {1}
}
child { node[] {$E(2,3)?$} 
    child { node[] {$0$} 
    edge from parent node[left,draw=none] {1}
	}
    child { node[] {$\left\lfloor\frac{n}{2}\right\rfloor$} 
    edge from parent node[right,draw=none] {0}
	}
    edge from parent node[right,draw=none] {0}
};
\end{tikzpicture}
\end{center}
\end{defi}
\end{frame}

\begin{frame}
\frametitle{Výsledky -- kapitola 2}
\begin{exam}
Ať $\EDGE_k$ je široká posloupnost sestávající z grafů s právě jednou hranou, potom
\[\lim_{F_{rud}}\EDGE_n\bbl(\exists x)(\exists y)E(x,y)\bbr=\0.\]
\end{exam}
\pause
\begin{itemize}
\item Další výsledky:
\begin{itemize}
\item Postačující podmínka na hustotu hran v $\G_n$ vedoucí k limitě bez hran\pause
\item Grafy co obsahují právě trojuhelník, čtverec, konstatní podgraf, \ldots $\to$ prázdná limita\pause
\item $\exists$ posloupnost s hustotou hran $\to 0$, ale limita obsahuje hranu\\
(Příklad 2.2.4, dotaz 3)
\end{itemize}
\end{itemize}
\end{frame}

\begin{frame}
\frametitle{Výsledky -- kapitola 3}
\begin{itemize}
\item Široká posloupnost obsahující právě velkou kliku: $\SK_k^{1/2}$
\item Široká posloupnost obsahující alespoň velkou kliku: $\CK_k^{1/2}$\pause
\item Teorie složitosti
\begin{itemize}
\item Pouze velká klika $\to$ polynomiální algoritmus ji najde
\item Alespoň velká klika $\to$ předpokládáme $\nexists$ polynomální algoritmus
\end{itemize}\pause
\item $G_{rud}$ -- limita druhého řádu\pause
\item Dokázal jsem
\begin{itemize}
\item $\SK_k^{1/2}\to_{G_{rud}}$ limita obsahuje relativně velkou kliku
\item $\CK_k^{1/2}\to_{G_{rud}}$ $\bbl\text{$\exists$ konečná klika}\bbr\not=\0$
\end{itemize}\pause
\item Domněnka: $\lim\CK_n^{1/2}$ neobsahuje kliku velikosti $\lfloor n/(c \ln n)\rfloor$
\end{itemize}
\end{frame}

\begin{frame}
\frametitle{Výsledky -- kapitola 4}
\begin{itemize}
\item Široká posloupnost cest s počátkem ve vrcholu $0$: $\pPATH_k$
\item $\pPATH_k \to_{F_{rud}}$ limita bez hran\pause
\item Třída funkcí $F_{nbtree}$\pause
\item Složitost $\TFNP$\pause
\item Výsledky:
\begin{itemize}
\item $\lim_{F_{nbtree}}\pPATH_n$ je cesta s počátkem, ale bez konce
\item $\Leftrightarrow$ funkce z $F_{nbtree}$ neumí najít konec cesty 
\item to odpovídá tomu, že $\PPA$ je ostrou nadmnožinou $\FP$ (oracle svět)
\end{itemize}
\end{itemize}
\end{frame}

\begin{frame}
\frametitle{Shrnutí}
\begin{itemize}
\item Výsledky z předchozích slidů $\approx$ analýza pravdivých sentencí\pause
\item V práci kombinuji metody teorie složitosti, kombinatoriky a teorie modelů\pause
\item Kromě úvodu, Preliminaries a části kapitoly $1$, jsou všechny výsledky vlastní\pause
\item Formuluji několik problémů navazujících na mé výsledky
\end{itemize}
\end{frame}

\begin{frame}
\frametitle{Nedostatky práce}
\begin{itemize}
\item Námitky oponenta\pause
\item Zřetelněji vyznačit autorství při prezentaci Forsingu v kapitole 1\pause
\item Ovšem: Chyby nemají za následek neplatnost nějakého tvrzení\pause
\item V erratě nepřesnosti a překlepy objasňuji
\end{itemize}
\end{frame}

\begin{frame}
\frametitle{Odpověď na otázku oponenta}
\begin{ques}
Mohl byste podrobněji ukázat, jak plyne nerovnost mezi (4.26) a (4.27)?
\end{ques}\pause
\begin{align*}
\prod_{i=0}^{n^{1/t}}\left(1-\frac{2}{n-2i-c-2}\right)&\geq 
\prod_{i=0}^{n^{1/t}}\left(1-\frac{2}{n-2n^{1/t}-c-2}\right)\\
&=\left (1-\frac{2}{n-2n^{1/t}-c-2}\right)^{n^{1/t}\textbf{+1}}\\
&\geq\left (1-\frac{2(n^{1/t}\textbf{+1})}{n-2n^{1/t}-c-2}\right)\\
\end{align*}
\vspace{-4em}
\pause
\begin{itemize}
\item Argument není ovlivněn\pause
\item BÚNO lze zvolit $T$ aby původní nerovnost platila
\end{itemize}
\end{frame}

\begin{frame}
\begin{center}
\huge
\text{Děkuji za pozornost}
\end{center}
\end{frame}

\end{document}
